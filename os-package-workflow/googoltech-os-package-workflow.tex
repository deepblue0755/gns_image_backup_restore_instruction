\documentclass[border=10pt]{standalone}

\usepackage{CJKutf8}
% ---- 中文支持(pdflatex)----

% ---- TikZ ----
\usepackage{tikz}
\usetikzlibrary{arrows.meta, positioning}

\begin{document}
\begin{CJK*}{UTF8}{gbsn}

\begin{tikzpicture}[
    node distance=2.8cm and 2.2cm,
    every node/.style={
        draw,
        rectangle,
        rounded corners,
        align=left,
        minimum width=4.2cm,
        minimum height=2.4cm,
        font=\small
    },
    arrow/.style={
        ->,
        thick,
        >=Stealth
    }
]

% ====== 上排 ======
\node (s1) {
\textbf{1. 需求确认}\\
\\
Redmine / 邮件\\
Issue 可追溯
};

\node (s2) [right=of s1] {
\textbf{2. 基准镜像}\\
选择与安装\\
\\
稳定版本\\
GNS 控制器完整安装
};

\node (s3) [right=of s2, minimum width=5.6cm] {
\textbf{3. 更新集成与版本管控}\\
\\
内核 / 驱动 / 组件\\
/opt 高频目录\\
Git 白名单追踪关键文件
};

\node (s4) [right=of s3] {
\textbf{4. 功能验证}\\
\\
功能 + 稳定性测试
};

% ====== 下排 ======
\node (s5) [below=of s4, minimum width=5.6cm] {
\textbf{5. 单元测试与一致性校验}\\
\\
googol\_release\_check.sh -c\\
自动重启 + 系统备份\\
上传打包服务器
};

\node (s6) [left=of s5, minimum width=5.0cm] {
\textbf{6. 镜像构建}\\
\\
U盘构建 / 自动化打包\\
按更新范围选择方式
};

\node (s7) [left=of s6, minimum width=5.0cm] {
\textbf{7. 镜像发布与交付}\\
\\
共享服务器发布\\
规范命名
};

\node (s8) [left=of s7] {
\textbf{8. 升级回退}\\
\\
U盘升级\\
原系统备份\\
快速回退
};

% ====== 箭头 ======
\draw[arrow] (s1) -- (s2);
\draw[arrow] (s2) -- (s3);
\draw[arrow] (s3) -- (s4);

\draw[arrow] (s4) -- ++(0,-1.4cm) -| (s5.north);

\draw[arrow] (s5) -- (s6);
\draw[arrow] (s6) -- (s7);
\draw[arrow] (s7) -- (s8);

\end{tikzpicture}

\end{CJK*}
\end{document}
