\documentclass[tikz,border=10pt]{standalone}

\usepackage{CJKutf8}
% ---- 中文支持(pdflatex)----

% ---- TikZ ----
\usepackage{tikz}
\usetikzlibrary{arrows.meta, positioning}
\usetikzlibrary{shapes.geometric, arrows.meta, positioning}

\begin{document}
\begin{CJK*}{UTF8}{gbsn}


\begin{tikzpicture}[
    node distance=22mm and 28mm,
    every node/.style={
        draw,
        rectangle,
        rounded corners,
        align=center,
        minimum width=42mm,
        minimum height=10mm
    },
    arrow/.style={
        ->,
        very thick,
        >=Stealth
    }
]

% --- First row (left -> right)
\node (pc) {Windows PC\\
自动连接远程 SAMBA};

\node (samba) [right=of pc] {SAMBA 共享目录\\
\texttt{$\backslash\backslash$192.168.201.73$\backslash$share\_mbr}};

\node (iso) [right=of samba] {上传 ISO\\
基础镜像};

% --- Second row (right -> left)
\node (codesys) [below=of iso] {上传 CODESYS 压缩包\\
\small (.tar.gz / .zip / .rar)};

\node (version) [left=of codesys] {生成 VERSION 文件\\
\small 8 位版本号\\(如 20220101)};

\node (flag) [left=of version] {生成触发标志文件\\
\texttt{I\_WANT\_TO\_UPDATE\_CODESYS}};

% --- Third row (downwards)
\node (service) [below=of flag] {服务器后台服务\\
自动触发递增打包};

\node (output) [right=of service] {生成新版本\\
递增打包镜像};

% --- Arrows
\draw[arrow] (pc) -- (samba);
\draw[arrow] (samba) -- (iso);

\draw[arrow] (iso) -- (codesys);
\draw[arrow] (codesys) -- (version);
\draw[arrow] (version) -- (flag);

\draw[arrow] (flag) -- (service);
\draw[arrow] (service) -- (output);

\end{tikzpicture}

\end{CJK*}
\end{document}
